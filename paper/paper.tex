% !TeX program = pdfLaTeX
\documentclass[smallextended]{svjour3}       % onecolumn (second format)
%\documentclass[twocolumn]{svjour3}          % twocolumn
%
\smartqed  % flush right qed marks, e.g. at end of proof
%
\usepackage{amsmath}
\usepackage{graphicx}
\usepackage[utf8]{inputenc}

\usepackage[hyphens]{url} % not crucial - just used below for the URL
\usepackage{hyperref}
\providecommand{\tightlist}{%
  \setlength{\itemsep}{0pt}\setlength{\parskip}{0pt}}

%
% \usepackage{mathptmx}      % use Times fonts if available on your TeX system
%
% insert here the call for the packages your document requires
%\usepackage{latexsym}
% etc.
%
% please place your own definitions here and don't use \def but
% \newcommand{}{}
%
% Insert the name of "your journal" with
% \journalname{myjournal}
%

%% load any required packages here





\begin{document}

\title{Open source landscape ecology tools \thanks{Grants or other notes
about the article that should go on the front page should be placed
here. General acknowledgments should be placed at the end of the
article.} }



\author{  Maximillian H.K. Hesselbarth \and  Jakub Nowosad \and  Author
3 \and  \ldots{} \and  }


\institute{
        Maximillian H.K. Hesselbarth \at
     Department of Ecosystem Modelling, University of Goettingen,
Buesgenweg 4, 37077 Goettingen, Germany \\
     \email{\href{mailto:maximilian.hesselbarth@uni-goettingen.de}{\nolinkurl{maximilian.hesselbarth@uni-goettingen.de}}}  %  \\
%             \emph{Present address:} of F. Author  %  if needed
    \and
        Jakub Nowosad \at
     Institute of Geoecology and Geoinformation, Adam Mickiewicz
University, Krygowskiego 10, 61-680 Poznan, Poland \\
     \email{\href{mailto:nowosad.jakub@gmail.com}{\nolinkurl{nowosad.jakub@gmail.com}}}  %  \\
%             \emph{Present address:} of F. Author  %  if needed
    \and
        Author 3 \at
     Adress author 3 \\
     \email{E-Mail author 3}  %  \\
%             \emph{Present address:} of F. Author  %  if needed
    \and
        \ldots{} \at
     \ldots{} \\
     \email{\ldots{}}  %  \\
%             \emph{Present address:} of F. Author  %  if needed
    \and
    }

\date{Received: date / Accepted: date}
% The correct dates will be entered by the editor


\maketitle

\begin{abstract}
max. (200 words) Abstract Abstract Abstract Abstract Abstract Abstract
Abstract Abstract Abstract Abstract Abstract Abstract Abstract Abstract
Abstract Abstract Abstract Abstract Abstract Abstract Abstract Abstract
Abstract Abstract Abstract Abstract Abstract Abstract Abstract Abstract
Abstract Abstract Abstract Abstract Abstract Abstract Abstract Abstract
Abstract Abstract Abstract Abstract Abstract Abstract Abstract Abstract
Abstract Abstract Abstract Abstract Abstract Abstract Abstract Abstract
Abstract Abstract Abstract Abstract Abstract Abstract Abstract Abstract
Abstract Abstract Abstract Abstract Abstract Abstract Abstract Abstract
Abstract Abstract Abstract Abstract Abstract Abstract Abstract Abstract
Abstract Abstract Abstract Abstract Abstract Abstract Abstract Abstract
Abstract Abstract Abstract Abstract Abstract Abstract Abstract Abstract
Abstract Abstract Abstract Abstract Abstract Abstract
\\
\keywords{
        keyword 1 \and
        keyword 2 \and
        keyword 3 \and
        keyword 4 \and
        \ldots{} \and
    }


\end{abstract}


\def\spacingset#1{\renewcommand{\baselinestretch}%
{#1}\small\normalsize} \spacingset{1}


\hypertarget{sec:intro}{%
\section{Introduction}\label{sec:intro}}

\hypertarget{sec:landscape_ecology}{%
\subsection{A short introduction to landscape
ecology}\label{sec:landscape_ecology}}

Landscape ecology focuses on how ecological processes are influenced and
modified by the heterogeneous landscapes they occur in and
simultaneously how the ecological processes themselves influence the
landscapes \cite{Turner1989,Turner2005,With2019}. In this context,
landscape ecology considers, besides others, spatial and temporal
dynamics of heterogeneous landscapes, interactions, fluxes and exchange
within these landscapes, how the landscapes influence ecological
processes (and vice versa) and lastly how to manage these heterogeneous
landscapes \cite{Risser1984,Turner1989}.

While human activities have altered their environment for millennials
\cite{Ellis2011}, in the past centuries the effects of humans on the
global environment have increased to an unknown high, known as the the
Anthropocene \cite{Crutzen2002}. Today, almost all ecosystems are
directly or indirectly influenced by human activities
\cite{Vitousek1997}. Thus, understanding the complex interactions
between landscapes and ecological processes also becomes more important
\cite{With2019}.

Because landscapes are defined as mosaics of different land covers,
ecosystems, habitat types, or land uses \cite{Forman1986,Forman1995},
spatial context is important and ecological processes will vary
spatially \cite{With2019}. Thus, in contrast to many other
sub-disciplines of ecology, landscape ecology emphasizes spatial
patterns \cite{Risser1984}. Consequently, the field of landscape ecology
relies on tools to preprocess, modify, model, analyze and visualize
spatial data.

\hypertarget{sec:open_source}{%
\subsection{Open-source software and R}\label{sec:open_source}}

Open-source software includes all software which is released under a
license that allows to freely use, modify and distribute the software
\cite{St.Laurent2008}. Open-source software development has many
advantages, such as fast innovation, transparency and reliability as
well as longevity due to many diverse contributors
\cite{vonKrogh2006,St.Laurent2008}.

One example of a successful open-source project is the R programming
language and its \emph{Comprehensive R Archive Network} (CRAN) for
extensions (also called packages) \cite{RCoreTeam2019}. Firstly
introduced in 1995 \cite{Smith2016}, today the programming language is
among the most popular programming languages, especially in ecology
\cite{Lai2019}.

\hypertarget{sec:R}{%
\subsection{Landscape ecology in R}\label{sec:R}}

\hypertarget{paragraph-headings}{%
\paragraph{Paragraph headings}\label{paragraph-headings}}

Use paragraph headings as needed.

\begin{align}
a^2+b^2=c^2
\end{align}

\bibliographystyle{spphys}
\bibliography{bibliography.bib}

\end{document}
